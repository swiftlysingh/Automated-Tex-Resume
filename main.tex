%%%%%%%%%%%%%%%%%%%%%%%%%%%%%%%%%%%%%%%
% Deedy - One Page Two Column Resume
% LaTeX Template
% Version 1.2 (16/9/2014)
%
% Original author:
% Debarghya Das (http://debarghyadas.com)
%
% Original repository:
% https://github.com/deedydas/Deedy-Resume
%
% IMPORTANT: THIS TEMPLATE NEEDS TO BE COMPILED WITH XeLaTeX
%
% This template uses several fonts not included with Windows/Linux by
% default. If you get compilation errors saying a font is missing, find the line
% on which the font is used and either change it to a font included with your
% operating system or comment the line out to use the default font.
% 
%%%%%%%%%%%%%%%%%%%%%%%%%%%%%%%%%%%%%%
% 
% TODO:
% 1. Integrate biber/bibtex for article citation under publications.
% 2. Figure out a smoother way for the document to flow onto the next page.
% 3. Add styling information for a "Projects/Hacks" section.
% 4. Add location/address information
% 5. Merge OpenFont and MacFonts as a single sty with options.
% 
%%%%%%%%%%%%%%%%%%%%%%%%%%%%%%%%%%%%%%
%
% CHANGELOG:
% v1.1:
% 1. Fixed several compilation bugs with \renewcommand
% 2. Got Open-source fonts (Windows/Linux support)
% 3. Added Last Updated
% 4. Move Title styling into .sty
% 5. Commented .sty file.
%
%%%%%%%%%%%%%%%%%%%%%%%%%%%%%%%%%%%%%%%
%
% Known Issues:
% 1. Overflows onto second page if any column's contents are more than the
% vertical limit
% 2. Hacky space on the first bullet point on the second column.
%
%%%%%%%%%%%%%%%%%%%%%%%%%%%%%%%%%%%%%%

\documentclass[]{deedy-resume-openfont}
\usepackage{fancyhdr}
 
\pagestyle{fancy}
\fancyhf{}
 
\begin{document}


%%%%%%%%%%%%%%%%%%%%%%%%%%%%%%%%%%%%%%
%
%     TITLE NAME
%
%%%%%%%%%%%%%%%%%%%%%%%%%%%%%%%%%%%%%%
\namesection{}{Pushpinder Pal Singh}{ 
\urlstyle{same}\href{https://bento.me/swiftlysingh}{bento/swiftlysingh} | \href{https://github.com/swiftlysingh}{github/swiftlysingh}\\
\href{mailto:sayhi@swiftlysingh.com}{sayhi@swiftlysingh.com} | +1 (916) 810-1905 | Sacramento, CA
}

%%%%%%%%%%%%%%%%%%%%%%%%%%%%%%%%%%%%%%
%     EDUCATION
%%%%%%%%%%%%%%%%%%%%%%%%%%%%%%%%%%%%%%

\section{Education} 

\runsubsection{MS in Computer Science |}
\descript{California State University}
\location{August 2024 - May 2026 (Expected) | Sacramento, California}

% \vspace{\topsep}
\sectionsep
\runsubsection{BE in Computer Engineering |}
\descript{Netaji Subhas University of Technology }
\location{Aug 2018 - May 2022 | New Delhi, IN}

%%%%%%%%%%%%%%%%%%%%%%%%%%%%%%%%%%%%%%
%     EXPERIENCE
%%%%%%%%%%%%%%%%%%%%%%%%%%%%%%%%%%%%%%

\section{Experience}
\textbf{\href{https://www.gojek.io/}{\runsubsection{Software Engineer - Mobile |}}} 
\descript{GOJEK}
\location{May 2022 – August 2024}
\begin{tightemize}
    \item Built a developer website (blog) and internal tools websites for Gojek, improving developer efficiency by 25\% and enhancing internal workflows.
    \item Co-developed the GoPay app in Flutter, supporting 130M daily users with an enhanced user experience.
    \item Implemented a Server Driven UI for GoPay's homepage, enabling dynamic updates without app releases.
    \item Integrated an in-house analytics framework, reducing third-party dependencies and cutting costs by 30\%.
\end{tightemize}
\sectionsep

\textbf{\href{https://summerofcode.withgoogle.com/projects/6623823417311232}{\runsubsection{Swift Developer |}}} 
\descript{Google Summer of Code - VideoLAN}
\location{June 2021 – August 2021}
\begin{tightemize}
    \item Added the "Continue Watching" feature in VLC media player for iOS, enhancing user retention and engagement.
    \item Improved UI/UX using Swift, aligning with Apple's design guidelines for a more intuitive user experience.
    \item Participated in code reviews, ensuring adherence to best practices and maintaining code quality.
    \item Fixed critical bugs, increasing the stability and performance of the VLC iOS app.
    \item Contributed to documentation, improving onboarding for new developers in the VideoLAN community.
\end{tightemize}

%%%%%%%%%%%%%%%%%%%%%%%%%%%%%%%%%%%%%%
%     SKILLS
%%%%%%%%%%%%%%%%%%%%%%%%%%%%%%%%%%%%%%

\section{Technical Skills }
\descript{Web Development (React), Apple Platforms (Swift), Flutter, Open Source, DevOPs, Self-hosting}

%%%%%%%%%%%%%%%%%%%%%%%%%%%%%%%%%%%%%%
%     Projects
%%%%%%%%%%%%%%%%%%%%%%%%%%%%%%%%%%%%%%

\section{Projects}

\textbf{\href{https://apps.apple.com/app/artiweather/id6446815662}{\runsubsection{ArtiWeather}}} 
\location{An Art Weather App}
\begin{tightemize}
    \item Developed and launched ArtiWeather for iOS 18, achieving 1.2k downloads and 5k organic impressions on launch day.
    \item Integrated on-device Stable Diffusion for weather-based image generation, solving real-time rendering challenges.
    \item Built widgets using WidgetKit for displaying weather visuals on the home screen.
    \item Currently optimizing SD Model for portrait images, enhancing download UX/UI, and adding support for custom cities.
\end{tightemize}
\sectionsep

\textbf{\href{https://github.com/swiftlysingh/Holder}{\runsubsection{Holder}}} 
\location{A Secure Card Vault}
\begin{tightemize}
    \item Built an iOS app using Swift for securely storing credit and debit card details.
    \item Used iOS Keychain with iCloud sync for encrypted storage, ensuring data privacy through local-only storage.
    \item Designed an intuitive interface with SwiftUI, providing a seamless user experience for managing card details.
    \item Integrated biometric authentication for added security, allowing easy and secure access to stored information.
\end{tightemize}

%%%%%%%%%%%%%%%%%%%%%%%%%%%%%%%%%%%%%
    % AWARDS
%%%%%%%%%%%%%%%%%%%%%%%%%%%%%%%%%%%%%

\section{Awards} 
\begin{tabular}{rll}
2021 & \textbf{Winner} & \textbf{Smart India Hackathon 2020} \\
     & & Built an AI IoT system to predict and adjust lighting and HVAC demand using sensors. \\
     & & SIH is a national hackathon by the Government of India, tackling real-world challenges.
\end{tabular}
\sectionsep

%%%%%%%%%%%%%%%%%%%%%%%%%%%%%%%%%%%%%%
%     PUBLICATIONS
%%%%%%%%%%%%%%%%%%%%%%%%%%%%%%%%%%%%%%

\section{Publications} 
\renewcommand\refname{\vskip -1.5em} % Couldn't get this working from the .cls file
\bibliographystyle{abbrv}
\bibliography{publications}
\nocite{*}

\end{document}  \documentclass[]{article}
