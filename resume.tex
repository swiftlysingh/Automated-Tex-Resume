% !TEX program = xelatex
\documentclass{fonts}
\usepackage{fancyhdr}
\pagestyle{fancy}
\fancyhf{}
\begin{document}


%%%%%%%%%%%%%%%%%%%%%%%%%%%%%%%%%%%%%%
%
%     TITLE NAME
%
%%%%%%%%%%%%%%%%%%%%%%%%%%%%%%%%%%%%%%
\namesection{}{Pushpinder Pal Singh}{ 
\urlstyle{same}\href{https://swifti.ng}{swifti.ng} | \href{https://github.com/swiftlysingh}{github/swiftlysingh} | \href{https://linkedin.com/in/pushpinderpalsingh}{linkedin} | \href{mailto:singh.pushpinderp@gmail.com}{singh.pushpinderp@gmail.com}
}

%%%%%%%%%%%%%%%%%%%%%%%%%%%%%%%%%%%%%%
%     SKILLS
%%%%%%%%%%%%%%%%%%%%%%%%%%%%%%%%%%%%%%

\section{Technical Skills}
\begin{tabular}{ @{} l @{\hspace{8pt}} p{13.5cm} }
\location{Languages} & Swift, Objective-C, Dart, Python \\
\location{Frameworks} & SwiftUI, UIKit, CoreML, Combine, WidgetKit, Flutter \\
\location{Architecture} & MVVM, VIPER, MVC, Server-Driven UI \\
\location{Tools} & Xcode, Git, CI/CD, TestFlight, App Store Connect, SPM, CocoaPods \\
\location{Testing} & XCTest, Unit Testing, UI Testing \\
\location{Other} & REST APIs, JSON, Protocol Buffers, async/await, Agile/Scrum \\
\end{tabular}

%%%%%%%%%%%%%%%%%%%%%%%%%%%%%%%%%%%%%%
%     EXPERIENCE
%%%%%%%%%%%%%%%%%%%%%%%%%%%%%%%%%%%%%%

\section{Experience}
\textbf{\runsubsection{Founding Mobile Engineer |}} 
\descript{Barkie.ai \datetext{Jun '25 – Oct '25}}
\begin{tightemize}
\item Architected the iOS app in \emphitalic{SwiftUI} with \emphitalic{MVVM}, delivering a HIG-compliant UI with \emphitalic{REST API} integration.
\item Engineered an on-device \emphitalic{CoreML} vision model using \emphitalic{async/await} to automate scorecard scanning via the GHIN API.
\item Implemented \emphitalic{CI/CD} pipelines for automated builds, \emphitalic{Unit Testing}, and beta deployments to \emphitalic{TestFlight}.
\end{tightemize}
\sectionsep

\textbf{\href{https://www.gojek.io/}{\runsubsection{Software Engineer – Mobile |}}} 
\descript{GOJEK \datetext{May '22 – Aug '24}}
\begin{tightemize}
\item Revamped Help Center UI using \emphitalic{UIKit} and \emphitalic{Combine}, improving usability and reducing support queries by \italictext{15\%}.
\item Built a Pin-based Auth SDK in \emphitalic{Swift} with \emphitalic{VIPER}, enabling unified authentication across GoTo apps.
\item Worked on GoPay app in \emphitalic{Flutter}, supporting \italictext{130M daily users}; conducted code reviews in \emphitalic{Agile} sprints.
\item Implemented \emphitalic{Server-Driven UI} for GoPay's homepage, enabling dynamic updates without app releases.
\item Integrated in-house analytics using \emphitalic{Protocol Buffers}, reducing third-party dependencies and cutting costs by \italictext{30\%}.
\end{tightemize}
\sectionsep

\textbf{\href{https://summerofcode.withgoogle.com/projects/6623823417311232}{\runsubsection{iOS Developer |}}} 
\descript{Google Summer of Code – VideoLAN \datetext{Jun '21 – Aug '21}}
\begin{tightemize}
\item Developed the ``Continue Watching'' feature in VLC for iOS using \emphitalic{Objective-C} and \emphitalic{UIKit}, enhancing user retention.
\item Participated in code reviews and fixed bugs in the legacy codebase, aligning UI with Apple's design guidelines.
\end{tightemize}
\sectionsep

\textbf{\runsubsection{Head of Open Source and Mobile Tech |}} 
\descript{IoSD \datetext{Jun '20 – Jun '21}}
\begin{tightemize}
\item Spearheaded development of a student mobile app in \emphitalic{Flutter}, scaling to over \italictext{10,000 daily active users}.
\item Delivered technical workshops on \emphitalic{Git} and open-source practices; mentored students contributing to major OSS projects.
\end{tightemize}

%%%%%%%%%%%%%%%%%%%%%%%%%%%%%%%%%%%%%%
%     OPEN SOURCE
%%%%%%%%%%%%%%%%%%%%%%%%%%%%%%%%%%%%%%

\section{Open Source Contributions}
\textbf{\runsubsection{\href{https://github.com/swiftlang}{SwiftLang |}}}
\descript{The Swift Programming Language – \linktext{https://github.com/swiftlang/swift-build/pulls?q=is:pr+author:swiftlysingh+}{GitHub} \datetext{Present}}
\begin{tightemize}
\item Implemented diagnostic flags within the Swift build system, improving error reporting and developer experience.
\item Enhanced compiler compatibility by adding checks for Apple Clang and removing unsupported flags.
\end{tightemize}

%%%%%%%%%%%%%%%%%%%%%%%%%%%%%%%%%%%%%%
%     Projects
%%%%%%%%%%%%%%%%%%%%%%%%%%%%%%%%%%%%%%

\section{Projects}
\textbf{\href{https://apps.apple.com/app/artiweather/id6446815662}{\runsubsection{ArtiWeather |}}} 
\descript{An Art Weather App – \linktext{https://apps.apple.com/app/artiweather/id6446815662}{App Store}}
\begin{tightemize}
\item Developed for iOS 18 using \emphitalic{SwiftUI} and \emphitalic{Combine}, achieving \italictext{1.2k downloads} and \italictext{5k impressions} on launch day.
\item Integrated on-device Stable Diffusion via \emphitalic{CoreML} for weather-based image generation with zero server costs.
\item Built home screen widgets using \emphitalic{WidgetKit} for displaying weather data and AI-generated artwork.
\end{tightemize}
\sectionsep

\textbf{\href{https://github.com/swiftlysingh/Holder}{\runsubsection{Holder |}}} 
\descript{A Secure Card Vault – \linktext{https://apps.apple.com/in/app/holder-a-secure-card-vault/id6475649492}{App Store}}
\begin{tightemize}
\item Built an iOS app in \emphitalic{Swift} and \emphitalic{SwiftUI} with iOS Keychain encryption, iCloud sync, and biometric authentication.
\item Shipped to \emphitalic{App Store Connect}; designed an intuitive card management UI with smooth animations and haptic feedback.
\end{tightemize}

%%%%%%%%%%%%%%%%%%%%%%%%%%%%%%%%%%%%%
%     AWARDS
%%%%%%%%%%%%%%%%%%%%%%%%%%%%%%%%%%%%%

\section{Awards} 
\textbf{\runsubsection{Smart India Hackathon 2020 |}}
\descript{Winner \datetext{2021}}
\begin{tightemize}
\item Built an AI-powered IoT system to predict and optimize lighting/HVAC demand; national hackathon by the Government of India.
\end{tightemize}

%%%%%%%%%%%%%%%%%%%%%%%%%%%%%%%%%%%%%%
%     EDUCATION
%%%%%%%%%%%%%%%%%%%%%%%%%%%%%%%%%%%%%%

\section{Education} 
\runsubsection{\makebox[5.6cm][l]{Master's in Computer Science} |}
\descript{California State University \datetext{Aug '24 – Expected May '26}}

\runsubsection{\makebox[5.6cm][l]{Technology Entrepreneurship} |}
\descript{Stanford University \datetext{Jun '25 – Aug '25}}

\runsubsection{\makebox[5.6cm][l]{BE in Computer Engineering} |}
\descript{Delhi University \datetext{Aug '18 – May '22}}

\end{document}
