% !TEX program = xelatex
\documentclass{fonts}
\usepackage{fancyhdr}
\pagestyle{fancy}
\fancyhf{}
\begin{document}


%%%%%%%%%%%%%%%%%%%%%%%%%%%%%%%%%%%%%%
%
%     TITLE NAME
%
%%%%%%%%%%%%%%%%%%%%%%%%%%%%%%%%%%%%%%
\namesection{}{Pushpinder Pal Singh}{ 
\urlstyle{same}\href{https://swifti.ng}{swifti.ng} | \href{https://github.com/swiftlysingh}{github/swiftlysingh} | \href{mailto:singh.pushpinderp@gmail.com}{singh.pushpinderp@gmail.com}
}

%%%%%%%%%%%%%%%%%%%%%%%%%%%%%%%%%%%%%%
%     EXPERIENCE
%%%%%%%%%%%%%%%%%%%%%%%%%%%%%%%%%%%%%%

\section{Experience}
\textbf{\href{https://www.gojek.io/}{\runsubsection{Software Engineer - Mobile |}}} 
\descript{GOJEK \datetext{May '22 – Aug '24}}
\begin{tightemize}
    \item Revamped Help Center UI using \emphitalic{UIKit}, improving usability and reducing support queries by \italictext{15\%}.
    \item Led crucial iOS initiatives for Help Center by collaborating seamlessly with Android, backend, and product teams.
    \item Built a Pin-based Auth SDK in \emphitalic{Swift} with \emphitalic{VIPER}, enabling a unified experience across GoTo apps
    \item Worked on the GoPay app in \emphitalic{Flutter}, supporting \italictext{130M daily users} with an enhanced user experience.
    \item Implemented a \emphitalic{Server Driven UI} for GoPay's homepage, enabling dynamic updates without app releases.
    \item Integrated in-house analytics framework utilising \emphitalic{ProtoBuffs}, reducing third-party dependencies and cutting costs by 30\%.
\end{tightemize}
\sectionsep

\textbf{\href{https://summerofcode.withgoogle.com/projects/6623823417311232}{\runsubsection{iOS Developer |}}} 
\descript{Google Summer of Code - VideoLAN \datetext{June '21 – Aug '21}}
\begin{tightemize}
    \item Worked on the "Continue Watching" feature in VLC media player for iOS, enhancing user retention and engagement.
    \item Worked with the \emphitalic{legacy codebase} and understanding \emphitalic{Objective-C} to fix bugs and implement features.
    \item Improved UI/UX using \emphitalic{UIKit}, aligning with Apple's design guidelines for a more intuitive user experience.
    \item Participated in code reviews, ensuring adherence to best practices and maintaining code quality.
    \item Contributed to documentation, improving onboarding for new developers in the VideoLAN community.
\end{tightemize}
\sectionsep

\textbf{\href{}{\runsubsection{Head of Open Source and Mobile Tech |}}} 
\descript{IoSD \datetext{June '20 – June '21}}
\begin{tightemize}
    \item Spearheaded the development of a student mobile app in Flutter, scaling it to over 10,000 daily active users.
    \item Designed and delivered technical workshops on Git and open-source best practices for first-year students.
    \item Mentored aspiring developers to contribute to major open-source projects, including the Linux kernel.
\end{tightemize}

%%%%%%%%%%%%%%%%%%%%%%%%%%%%%%%%%%%%%%
%     TECHNICAL CONTRIBUTIONS
%%%%%%%%%%%%%%%%%%%%%%%%%%%%%%%%%%%%%%

\section{Open Source}

\textbf{\runsubsection{\href{https://github.com/swiftlang}{SwiftLang |}}}
\descript{The Swift programming language - \linktext{https://github.com/swiftlang/swift-build/pulls?q=is:pr+author:swiftlysingh+}{GitHub} \datetext{Present}}
\begin{tightemize}
    \item Implemented new diagnostic flags within the Swift build system, improving error reporting and developer experience.
    \item Enhanced compiler compatibility by adding checks for Apple Clang and removing unsupported flags.
    \item Contributed to Swift Information Architecture Project, unifying content and improving documentation accessibility.
    \item Provided support and guidance to the Swift community on the forums, fostering collaboration and knowledge sharing.
\end{tightemize}
\sectionsep

%%%%%%%%%%%%%%%%%%%%%%%%%%%%%%%%%%%%%%
%     SKILLS
%%%%%%%%%%%%%%%%%%%%%%%%%%%%%%%%%%%%%%

% \section{Technical Skills }
% \descript{Apple Platforms (Swift), Flutter, Open Source, DevOPs, Self-hosting, IoT, 3D Printing}

%%%%%%%%%%%%%%%%%%%%%%%%%%%%%%%%%%%%%%
%     Projects
%%%%%%%%%%%%%%%%%%%%%%%%%%%%%%%%%%%%%%

\section{Projects}

\textbf{\href{https://apps.apple.com/app/artiweather/id6446815662}{\runsubsection{ArtiWeather |}}} 
\descript{An Art Weather App - \linktext{https://apps.apple.com/app/artiweather/id6446815662}{AppStore}}
\begin{tightemize}
    \item Developed ArtiWeather for iOS 18 using \emphitalic{SwiftUI}, achieving \italictext{1.2k downloads} and \italictext{5k organic impressions} on launch day.
    \item Integrated \emphitalic{on-device Stable Diffusion} using \emphitalic{CoreML} for weather-based image generation, with no server costs.
    \item Built beautiful widgets using \emphitalic{WidgetKit} for displaying weather and generated images on the home screen.
    \item Working on optimizing SD Model, enhancing download UX/UI, and adding support for additional cities.
\end{tightemize}
\sectionsep

\textbf{\href{https://github.com/swiftlysingh/Holder}{\runsubsection{Holder |}}} 
\descript{A Secure Card Vault - \linktext{https://apps.apple.com/in/app/holder-a-secure-card-vault/id6475649492}{AppStore}}
\begin{tightemize}
    \item Built an iOS app using Swift and SwiftUI for securely storing credit and debit card details.
    \item Used iOS Keychain with iCloud sync for encrypted storage, ensuring data privacy through local-only storage.
    \item Designed an intuitive interface with SwiftUI, providing a seamless user experience for managing card details.
    \item Integrated biometric authentication for added security, allowing easy and secure access to stored information.
\end{tightemize}

%%%%%%%%%%%%%%%%%%%%%%%%%%%%%%%%%%%%%
    % AWARDS
%%%%%%%%%%%%%%%%%%%%%%%%%%%%%%%%%%%%%

\section{Awards} 
\textbf{\runsubsection{Smart India Hackathon 2020 |}}
\descript{Winner \datetext{2021}}
\begin{tightemize}
    \item Built an AI IoT system to predict and adjust lighting and HVAC demand using sensors.
    \item SIH is a national hackathon by the Government of India, tackling real-world challenges.
\end{tightemize}
\sectionsep

%%%%%%%%%%%%%%%%%%%%%%%%%%%%%%%%%%%%%%
%     PUBLICATIONS
%%%%%%%%%%%%%%%%%%%%%%%%%%%%%%%%%%%%%%

% Comment out original bibliography approach
% \renewcommand\refname{\vskip -1.5em} 
% \bibliographystyle{abbrv}
% \bibliography{publications}
% \nocite{*}

%%%%%%%%%%%%%%%%%%%%%%%%%%%%%%%%%%%%%%
%     EDUCATION
%%%%%%%%%%%%%%%%%%%%%%%%%%%%%%%%%%%%%%

\section{Education} 
\runsubsection{\makebox[5.6cm][l]{Technology Entrepreneurship} |}
\descript{Stanford University  \datetext{June'25 - Aug '25}}

\sectionsep

\runsubsection{\makebox[5.6cm][l]{Master's in Computer Science} |}
\descript{California State University \datetext{Aug '24 - May '26}}

\sectionsep

\runsubsection{\makebox[5.6cm][l]{BE in Computer Engineering} |}
\descript{Delhi University  \datetext{Aug '18 - May '22}}

\end{document}
