\documentclass{fonts}
\usepackage{fancyhdr}
\pagestyle{fancy}
\fancyhf{}
\begin{document}


%%%%%%%%%%%%%%%%%%%%%%%%%%%%%%%%%%%%%%
%
%     TITLE NAME
%
%%%%%%%%%%%%%%%%%%%%%%%%%%%%%%%%%%%%%%
\namesection{}{Pushpinder Pal Singh}{ 
\urlstyle{same}\href{https://swifti.ng}{swifti.ng} | \href{https://github.com/swiftlysingh}{github/swiftlysingh} | \href{mailto:pushpinderpal19@gmail.com}{pushpinderpal19@gmail.com}
}

%%%%%%%%%%%%%%%%%%%%%%%%%%%%%%%%%%%%%%
%     EXPERIENCE
%%%%%%%%%%%%%%%%%%%%%%%%%%%%%%%%%%%%%%

\section{Experience}
\textbf{\href{https://www.gojek.io/}{\runsubsection{Software Engineer - Mobile |}}} 
\descript{GOJEK \datetext{May '22 – Aug '24}}
\begin{tightemize}
    \item Revamped Help Center UI, improving usability and reducing support queries by 15\%.
    \item Built a PIN-based auth framework, enabling a single PIN across multiple GoTo apps, reducing development time by 20\%.
    \item Co-developed the GoPay app in Flutter, supporting 130M daily users with an enhanced user experience.
    \item Implemented a Server Driven UI for GoPay's homepage, enabling dynamic updates without app releases.
    \item Integrated an in-house analytics framework, reducing third-party dependencies and cutting costs by 30\%.
\end{tightemize}
\sectionsep

\textbf{\href{https://summerofcode.withgoogle.com/projects/6623823417311232}{\runsubsection{iOS Developer |}}} 
\descript{Google Summer of Code - VideoLAN \datetext{June '21 – Aug '21}}
\begin{tightemize}
    \item Added the "Continue Watching" feature in VLC media player for iOS, enhancing user retention and engagement.
    \item Improved UI/UX using Swift, aligning with Apple's design guidelines for a more intuitive user experience.
    \item Participated in code reviews, ensuring adherence to best practices and maintaining code quality.
    \item Fixed critical bugs, increasing the stability and performance of the VLC iOS app.
    \item Contributed to documentation, improving onboarding for new developers in the VideoLAN community.
\end{tightemize}

%%%%%%%%%%%%%%%%%%%%%%%%%%%%%%%%%%%%%%
%     TECHNICAL CONTRIBUTIONS
%%%%%%%%%%%%%%%%%%%%%%%%%%%%%%%%%%%%%%

\section{Open Source}

\textbf{\runsubsection{\href{https://github.com/swiftlang}{SwiftLang |}}}
\descript{The Swift programming language - \linktext{https://github.com/swiftlang/swift-build/pulls?q=is:pr+author:swiftlysingh+}{GitHub} \datetext{Present}}
\begin{tightemize}
    \item Added new flags for diagnostic groups in the build system.
    \item Added a check for the Apple Clang and removing unsported flags.
    \item Swift Information Architecture Projects
    \item Supporting people on the forums
\end{tightemize}
\sectionsep

\textbf{\runsubsection{VideoLAN |}}
\descript{The VLC media player \datetext{2021}}
\begin{tightemize}
    \item Redesigned video controls UI to match iOS design guidelines, increasing user satisfaction by 27\%.
    \item Optimized media playback for low-powered devices, reducing battery consumption by 18\%.
    \item Contributed to accessibility improvements, bringing VoiceOver support to 95\% of app functions.
\end{tightemize}
\sectionsep

%%%%%%%%%%%%%%%%%%%%%%%%%%%%%%%%%%%%%%
%     SKILLS
%%%%%%%%%%%%%%%%%%%%%%%%%%%%%%%%%%%%%%

% \section{Technical Skills }
% \descript{Apple Platforms (Swift), Flutter, Open Source, DevOPs, Self-hosting, IoT, 3D Printing}

%%%%%%%%%%%%%%%%%%%%%%%%%%%%%%%%%%%%%%
%     Projects
%%%%%%%%%%%%%%%%%%%%%%%%%%%%%%%%%%%%%%

\section{Projects}

\textbf{\href{https://apps.apple.com/app/artiweather/id6446815662}{\runsubsection{ArtiWeather |}}} 
\descript{An Art Weather App - \linktext{https://apps.apple.com/app/artiweather/id6446815662}{AppStore}}
\begin{tightemize}
    \item Developed and launched ArtiWeather for iOS 18, achieving 1.2k downloads and 5k organic impressions on launch day.
    \item Integrated on-device Stable Diffusion for weather-based image generation, solving real-time rendering challenges.
    \item Built widgets using WidgetKit for displaying weather visuals on the home screen.
    \item Currently optimizing SD Model for portrait images, enhancing download UX/UI, and adding support for custom cities.
\end{tightemize}
\sectionsep

\textbf{\href{https://github.com/swiftlysingh/Holder}{\runsubsection{Holder |}}} 
\descript{A Secure Card Vault - \linktext{https://apps.apple.com/in/app/holder-a-secure-card-vault/id6475649492}{AppStore}}
\begin{tightemize}
    \item Built an iOS app using Swift for securely storing credit and debit card details.
    \item Used iOS Keychain with iCloud sync for encrypted storage, ensuring data privacy through local-only storage.
    \item Designed an intuitive interface with SwiftUI, providing a seamless user experience for managing card details.
    \item Integrated biometric authentication for added security, allowing easy and secure access to stored information.
\end{tightemize}

%%%%%%%%%%%%%%%%%%%%%%%%%%%%%%%%%%%%%
    % AWARDS
%%%%%%%%%%%%%%%%%%%%%%%%%%%%%%%%%%%%%

\section{Awards} 
\textbf{\runsubsection{Smart India Hackathon 2020 |}}
\descript{Winner \datetext{2021}}
\begin{tightemize}
    \item Built an AI IoT system to predict and adjust lighting and HVAC demand using sensors.
    \item SIH is a national hackathon by the Government of India, tackling real-world challenges.
\end{tightemize}
\sectionsep

%%%%%%%%%%%%%%%%%%%%%%%%%%%%%%%%%%%%%%
%     PUBLICATIONS
%%%%%%%%%%%%%%%%%%%%%%%%%%%%%%%%%%%%%%

% Comment out original bibliography approach
% \renewcommand\refname{\vskip -1.5em} 
% \bibliographystyle{abbrv}
% \bibliography{publications}
% \nocite{*}

%%%%%%%%%%%%%%%%%%%%%%%%%%%%%%%%%%%%%%
%     EDUCATION
%%%%%%%%%%%%%%%%%%%%%%%%%%%%%%%%%%%%%%

\section{Education} 

\runsubsection{MS in Computer Science |}
\descript{California State University, Sacramento \datetext{Aug '24 - May '26}}

\sectionsep
\runsubsection{BE in Computer Engineering |}
\descript{Netaji Subhas University of Technology \datetext{Aug '18 - May '22}}

\end{document}
